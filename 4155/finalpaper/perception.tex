\section{Perception \& Computer Vision}
\label{section::perception}

The OpenCV library was used to interface with the webcam. The latest major build of OpenCV stores image data as numpy arrays. This allows for simplified processing and iterating through the image data to find values matching a certain criteria. All images were blurred with Gaussian filters before any other image processing took place. Initially, four areas of the image were selected: the front and back colours on the robot, the colour of the goal and the colour of the table. This took a small average of Hue, Saturation, and Value (HSV) values in the pixels surrounding the user’s selected area, the tolerance window was set to accept values within range of this average when later searching for areas of the image that matched. Instead of looking for objects to avoid, and the edge of the table, the search was for the table values. To determine if an area is safe to move, a check was completed to determine if most of that area matches the HSV values of the table's surface. 