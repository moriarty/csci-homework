\section{Section 1}
\subsection{Describe Moral Agents Involved}
This case revolves around three moral agents. ``Screen Keyboard Inc.'' (SKI) is the creator of the proprietary, and expensive software ``ScreenMouse.'' The University, who has created Free and Open Source (FROS) software offering the same features: ``KeyMouse''. And all disabled peoples who stand to benefit from access to such software and whose quality of life could be improved.
\subsection{Identify the Moral Aspects}
Of the many aspects that are involved, we shall focus only on a selection. 

Discrimination: SKI has priced their software extremely high. They’re taking advantage of two facts. First, insurance will likely cover the cost. Second, the value of being able to use a computer, and complete tasks which an abled bodied people can, is high. Their business model directly targets an audience who is covered by health care insurance, and they are taking advantage of peoples who are in need.

Monopolization: Allowing one company to control the market results in the products and prices of the market being set at the discretion of the company. In this case SKI does not want to be undermined by a free software alternative. However if there were no alternative to ``ScreenMouse'', the price and new features being added would be up to the company, being a company they would act primarily to achieve their own capitalist goals. FROS allows for the rapid addition of features, and customizations. However FROS has its downsides, usually in the customer support. Issues are tracked, and feedback is constantly monitored to improve the product for the next release, but you cannot phone a representative when it doesn’t work. This feature is often worth the extra cost, and the availability of FROS keeps the proprietary software fees lower. 
\subsection{State Ethical Implications}
The availability of FROS forces SKI to rethink their business plan. Considering how long Ski took to develop ``ScreenMouse'', their targeted audience and pricing strategies, they could be slow to adapt to this new competitor. Ultimately, they could go bankrupt, resulting in a loss of jobs and money for those involved with SKI. However, allowing SKI control over this market restricts the options of disabled peoples. The marketing strategies in place at SKI add strain on the Health Care and Insurance industries. Whether public, private or tiered heath insurance will cover the costs does not allow SKI the right to raise the price arbitrarily high. There are added costs to being disabled: physically, mentally and monetary. The more corporate greed tries to take advantage of peoples in need the more unaffordable their lives become; adding to the obstacles they already must overcome.
\subsubsection{State Choices \& Consequences}
There exist two options:
\begin{enumerate}
\item 
Allow FROS \& competition. The competition could be other FROS projects or paid software. The benefit to the other products available on the market is lower prices. Companies competing for business would increasingly add features and having the FROS alternatives insures that the software remains accessible to everyone, regardless of financial means. The open source would allow for continued development and increased level of customization, further improving the lives of the users. 
\item 
Allow SKI a monopoly. This restricts innovation to the business plan of SKI, allowing them to set the prices, decide the features and reap all of the financial benefits. This model helps a smaller population, and adds strain on the disabled population and the support systems in place, which SKI expects to pay for the software.
\end{enumerate}