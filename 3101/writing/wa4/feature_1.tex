\section{Read Chapter 10 ``Ethical Interest in Free and Open Source Software'' from \textit{The Handbook of Information and Computer Ethics}}
\subsection{Find an Ethical arguement in the reading and critically discuss it.}

\begin{center}
``[Free Software] is the morally superior to [Open Source Software].''
\end{center}

% State the arguement
% Discuss weather you think the arguement is plausible. 
The argument stated above is given as the conclusion of points made by Chopra and Dexter. Although I agree that this is plausible, I disagree in general. This concluding argument is based on morals and it can be argued to go either way depending on ones moral beliefs. In the sense the terms are presented in this article, it's analogous to comparing political views, in fact, the political compass could be used to represent software well. As a active contributor to OSS and Free and Open Source Software FOSS, I believe the attempt to liberate software, placed on Free Software Foundation goes too far. Stallman attempted to liberate software yet the FSF licensing terms are so tight that they seem authoritarian. If we view OSS as Stallman does: ``Open source is a development methodology; free software is a social movement.'' [Stallman], then we can't compare which one is morally superior; one's an apple and the other an orange. The Open Source Initiative disagrees with Stallman on one main point: Stallman believes non-FS to be unethical. Stallman's involvement in creating social change and disregarding the views of others is a move towards the software equivalent of communism. Although communism in theory could work, it hasn't yet been properly implemented. I do not believe the FSF can be morally superior if it disregards all other views.