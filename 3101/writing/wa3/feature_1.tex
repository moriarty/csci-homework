\section{Reread Michael Bayles ``The Professions'' from \textit{Professions and Professionalization}}
\subsection{Explain in own words one feature from one of the three sets}


\begin{center}
``Organizations to which professionals belong, as members, claim to represent their members and the interest of their professions.''
\end{center}
Professional organizations are involved in many aspects of the lives of the members which they represent. Once an engineer graduates with a degree from an accredited institution, they are not yet a professional engineer. The process for moving from B.Eng to P.Eng is governed by an association of professional engineers. In Nova Scotia, this group is ``Association of Professional Engineers Nova Scotia'' (APENS). Gaining your bachelor degree from Dalhousie University does not guarantee you will become a Professional Engineer. To ensure high standards in engineering, you must first work under the supervision of a P.Eng in the field. As mentioned in the text, the organizations are not always open to all members of the profession. These organizations do not represent people with a common degree, they represent members who they see to be included in their profession. This keeps the monetary value of the profession high, which is of interest to all represented members. They are also involved in accrediting institutions, which means they can control the supply and further increase the value of their members. At the same time they offer exclusive benefits to their members, such as scholarships for their children.

